\documentclass[hyperref={pdfpagelabels=false},xcolor={dvipsnames},compress]{beamer}
\usepackage{lmodern}
\usepackage{listings}
\lstdefinelanguage{Kotlin}{
    comment=[l]{//},
    commentstyle={\color{gray}\ttfamily},
    emph={filter, first, firstOrNull, forEach, lazy, map, mapNotNull, println, until, downTo, TODO},
    emphstyle={\color{OrangeRed}\ttfamily},
    keywords={!in, !is, abstract, actual, annotation, as, as?, break, by, catch, value, class, companion, const, constructor, continue, crossinline, data, delegate, do, dynamic, else, enum, expect, external, false, field, file, final, finally, for, fun, get, if, import, in, infix, init, inline, inner, interface, internal, is, lateinit, noinline, null, object, open, operator, out, override, package, param, private, property, protected, public, receiveris, reified, return, return@, sealed, set, setparam, super, suspend, tailrec, this, throw, true, try, typealias, typeof, val, var, vararg, when, where, while},
    keywordstyle={\color{NavyBlue}\bfseries\ttfamily},
    morecomment=[s]{/*}{*/},
    morestring=[b]",
    morestring=[s]{"""*}{*"""},
    ndkeywords={@Deprecated, @JvmField, @JvmInline, @JvmName, @JvmOverloads, @JvmStatic, @JvmSynthetic, Array, Byte, UByte, Double, Float, Int, UInt, Integer, Iterable, Long, Runnable, Short, String, Any, Unit, Nothing},
    ndkeywordstyle={\color{BurntOrange}\ttfamily},
    sensitive=true,
    stringstyle={\color{ForestGreen}\ttfamily},
    identifierstyle={\ttfamily},
    basicstyle={\ttfamily},
    numberstyle={\ttfamily},
}
\lstset{language=Kotlin,showstringspaces=false,basicstyle=\small\ttfamily}
\usetheme{Berlin}

\AtBeginSection[]{
    \begin{frame}
        \vfill
        \centering
        \begin{beamercolorbox}[sep=8pt,center,shadow=false,rounded=false]{title}
            \usebeamerfont{title}\insertsectionhead\par%
        \end{beamercolorbox}
        \vfill
    \end{frame}
}

\title{Kotlin - aimed to make developers happier}
\author{Maciej Procyk}
\date{\includegraphics[width=75pt]{images/happy.png}\\November 15, 2021}
\begin{document}

    \logo{\includegraphics[width=40pt]{images/logo.png}}

    {
        \setbeamertemplate{logo}{}
        \setbeamertemplate{headline}{}
        \begin{frame}
            \maketitle
        \end{frame}
    }


    \begin{frame}
        \frametitle{Table of contents}
        \tableofcontents
    \end{frame}


    \section{Introduction}

    \subsection{Hello Kotlin}

    \begin{frame}[fragile]{Say "Hello World" and go home}

        \begin{exampleblock}{HelloWorld.kt}
            \begin{lstlisting}
fun main() = println("Hello, Kotlin World")
            \end{lstlisting}
        \end{exampleblock}
    \end{frame}

    \subsection{Main ideas}

    \begin{frame}{Short history background}
        \begin{itemize}
            \uncover<1->{\item unveiled in 2011 (being designed from 2010) by JetBrains}
            \uncover<2->{\item goal: compile as quickly as Java but offer as rich features as Scala}
            \uncover<3->{\item \alt<3>{you guess a name source (think of Java)}{named for the island Kotlin near Saint Petersburg}}
            \uncover<5->{\item announced as official Android language in 2017}
            \uncover<6->{\item becomes multiplatform programming language}
        \end{itemize}
    \end{frame}

    \begin{frame}{Main design goals}
        \begin{itemize}
            \item \uncover<1->{concise, object-oriented language}\uncover<2->{, "better" than Java} \uncover<3->{but fully operable with it}
            \uncover<4->{\item enable fast turnaround for developer}
            \uncover<5->{\item provide many ways to reuse your code}
            \uncover<6->{\item expressiveness - new syntax features}
            \uncover<7->{\item scalability - introduce coroutines for JVM}
            \uncover<8->{\item enable systems migration step-by-step}
            \uncover<9->{\item provide great tooling and easy learning curve}
        \end{itemize}
    \end{frame}

    \begin{frame}{Use cases}
        \begin{itemize}
            \uncover<1->{\item main Android language}
            \uncover<2->{\item server-side development}
            \uncover<3->{\item multiplatform (mobile) programming}
            \uncover<4->{\item frontend development}
            \uncover<5->{\item being introduced to data science}
            \uncover<6->{\item even tried in competitive programming}
        \end{itemize}
    \end{frame}


    \section{Language}

    \subsection{Introduction}

    \begin{frame}[fragile]{"Simplify" our greeting}
        \begin{exampleblock}
            <1->{HelloWorldKt.kt}
            \begin{lstlisting}
object HelloWorldKt {
  @JvmStatic
  fun main(args: Array<String>): Unit {
      val name: String? = readLine()
      if (name == null) return
      System.out.println("Hello " + name)
  }
}
            \end{lstlisting}
        \end{exampleblock}
    \end{frame}

    \begin{frame}[fragile]{Imperative approach}
        \begin{exampleblock}
            <1->{HelloWorld.kt}
            \begin{lstlisting}
fun main() {
  val name = readLine() ?: return
  println("Hello $name")
}
            \end{lstlisting}
        \end{exampleblock}
    \end{frame}

    \begin{frame}[fragile]{Declarative approach}
        \begin{exampleblock}
            <1->{HelloWorld.kt}
            \begin{lstlisting}
fun main() = readLine()
  ?.let { println("Hello $it") }
  ?: Unit
            \end{lstlisting}
        \end{exampleblock}
    \end{frame}

    \begin{frame}{Basic language assumptions}
        \begin{itemize}
            \item \uncover<1->{items are }\uncover<2->{\texttt{public}}\uncover<1->{ by default}
            \item \uncover<3->{classes and methods are }\uncover<4->{\texttt{final}}\uncover<3->{ by default}
            \item \uncover<5->{there is no }\uncover<6->{\texttt{static}}\uncover<5->{ concept in mind}
            \uncover<7->{(but the abstraction of \texttt{companion object} exists)}
            \item \uncover<9->{emphasis of }\uncover<10->{null-ability}\uncover<9->{ for every type }\uncover<10->{with \texttt{?}}
            \item \uncover<11->{no primitive types}\uncover<12->{ but quite a lot of optimizations including effective usage of them}
            \item \uncover<13->{\texttt{final} variables used by default}\uncover<14->{ with the keyword \texttt{val}}
            \item \uncover<15->{high usage of type inference in different contexts (\texttt{val}s and \texttt{var}s used as fields and variables definitions)}
        \end{itemize}
    \end{frame}

    \begin{frame}{Program structure}
        \begin{itemize}
            \item functions as first-class citizens (functional types of type \texttt{(A, B) -> C}) \pause
            \item not only \texttt{class}es and \texttt{interface}s, but also \texttt{object}s (and \texttt{sealed} versions of them) and \texttt{value class}es \pause
            \item no \texttt{package private} visibility but \texttt{public}, \texttt{protected}, \texttt{private} and \texttt{internal} are used \pause
            \item statically typed variables with: \pause
            \begin{itemize}
                \item number types like \texttt{Byte}, \texttt{Short}, \texttt{Int}, \texttt{Long}, (with unsigned versions) \texttt{Float}, \texttt{Double} (all with explicit conversion) \pause
                \item \texttt{String}, \texttt{Char} and \texttt{Boolean} types \pause
                \item \texttt{Array<T>} and specialized versions for primitives \texttt{IntArray}, \texttt{DoubleArray}, \ldots \pause
                \item ends on \texttt{Any} and includes also \texttt{null}able types
            \end{itemize}
        \end{itemize}
    \end{frame}

    \begin{frame}[fragile]{Null safety - elvis operator}
        \begin{exampleblock}{ElvisOperator.kt}
            \begin{onlyenv}<1->
                \begin{lstlisting}
val beOrNot: String? = "be"
                \end{lstlisting}
            \end{onlyenv}
            \begin{onlyenv}<2->
                \begin{lstlisting}
val answerSize: Int? = beOrNot?.length
                \end{lstlisting}
            \end{onlyenv}
            \begin{onlyenv}<3->
                \begin{lstlisting}
val decided: String = beOrNot ?: "not"
                \end{lstlisting}
            \end{onlyenv}
            \begin{onlyenv}<4->
                \begin{lstlisting}
val sureAnswerSize: Int = beOrNot!!.length
                \end{lstlisting}
            \end{onlyenv}
        \end{exampleblock}

        \begin{onlyenv}<5->
            also supported in safe casting with \texttt{as?} operator
        \end{onlyenv}
    \end{frame}

    \begin{frame}[fragile]{Lambdas definitions}
        Kotlin introduces new syntax for lamdas i.e. $\lbrace$ $\rbrace$
        \begin{exampleblock}{Lambdas.kt}
            \begin{onlyenv}<1->
                \begin{lstlisting}
val hasSense: (Int) -> Boolean = { it == 42 }
                \end{lstlisting}
            \end{onlyenv}
            \begin{onlyenv}<2->
                \begin{lstlisting}
val beep = { println("beep") }
                \end{lstlisting}
            \end{onlyenv}
            \begin{onlyenv}<3->
                \begin{lstlisting}
val printWarn = { msg: Any -> 
  println("WARN: $msg")
}
                \end{lstlisting}
            \end{onlyenv}
            \begin{onlyenv}<4->
                \begin{lstlisting}
beep(); printWarn(hasSense(42))
                \end{lstlisting}
            \end{onlyenv}
        \end{exampleblock}
    \end{frame}

    \begin{frame}[fragile]{Expression controlled flow}
        \begin{exampleblock}{FlowFeatures.kt}
            \begin{onlyenv}<1->
                \begin{lstlisting}
val state = if (readLine() == "42") "alive"
            else "dead"

                \end{lstlisting}
            \end{onlyenv}
            \begin{onlyenv}<2->
                \begin{lstlisting}
val mood =
  when (val grade = (2..6).random()) {
    in 3..4 -> "uff... I passed with $grade"
    5, 6 -> "happy getting $grade"
    else -> "sad because of $grade"
  }
                \end{lstlisting}
            \end{onlyenv}
        \end{exampleblock}
    \end{frame}

    \begin{frame}[fragile]{Named loops and scopes}
        \begin{exampleblock}{FlowFeatures.kt}
            \begin{onlyenv}<1->
                \begin{lstlisting}
val friends = listOf("Alice", "Bob", "Carol")

for (friend in friends)
  println("Hello $friend")

friends@ for (friend in friends) {
  for (i in 0..3) {
    println("Hello $friend")
    if (friend.last() == 'a') break@friends
  }
}
                \end{lstlisting}
            \end{onlyenv}
        \end{exampleblock}
    \end{frame}

    \begin{frame}[fragile]{Fast return from flow}
        \begin{exampleblock}{FlowFeatures.kt}
            \begin{onlyenv}<1-1>
                \begin{lstlisting}
while (true) {
  readLine()
    ?.let(::println)
    ?: break
}
                \end{lstlisting}
            \end{onlyenv}
            \begin{onlyenv}<2-2>
                \begin{lstlisting}
while (true) readLine()?.let(::println) ?: break
                \end{lstlisting}
            \end{onlyenv}
            \begin{onlyenv}<3->
                \begin{lstlisting}
while (true) {
  readLine()
    ?.let(::println)
    ?: break
}

val name = readLine() ?: return
val surname = readLine() ?: return
println("Your full name is $name $surname")
                \end{lstlisting}
            \end{onlyenv}
        \end{exampleblock}
    \end{frame}

    \subsection{Functions}

    \begin{frame}[fragile]{Functions in Kotlin}
        \begin{exampleblock}{BasicFunctions.kt}
            \begin{onlyenv}<1->
                \begin{lstlisting}
fun add(a: Int, b: Int): Int {
  return a + b
}

fun sub(a: Int, b: Int) = a - b
                \end{lstlisting}
            \end{onlyenv}
            \begin{onlyenv}<2->
                \begin{lstlisting}
val op = ::add
                \end{lstlisting}
            \end{onlyenv}
        \end{exampleblock}
        \uncover<2->{Kotlin also supports functions references that belong to one of the \texttt{KFunction<out R>} subtypes}
    \end{frame}

    \begin{frame}[fragile]{Default arguments}
        \begin{exampleblock}{SyntaxFeaturesArgs.kt}
            \begin{onlyenv}<1->
                \begin{lstlisting}
fun fib(
  idx: UInt,
  curr: Long = 1,
  last: Long = 0,
): Long =
  if (idx == 0U) curr
  else fib(idx - 1U, curr + last, curr)

fib(idx = 42U, curr = 42, last = 1)
fib(42U)
                \end{lstlisting}
            \end{onlyenv}
        \end{exampleblock}
    \end{frame}

    \begin{frame}[fragile]{Extension functions}
        \begin{exampleblock}{ExtensionFunctions.kt}
            \begin{onlyenv}<1->
                \begin{lstlisting}
// this can be omitted
fun Any.printHash() =
  println(hashCode())

// also for final classes
fun String.twice() = this + this

"42".twice().printHash()
                \end{lstlisting}
            \end{onlyenv}
        \end{exampleblock}
        Defined as \texttt{static} functions with first hidden parameter desugared from \texttt{this}
    \end{frame}

    \begin{frame}[fragile]{Infix functions}
        \begin{exampleblock}{InfixFunctions.kt}
            \begin{onlyenv}<1->
                \begin{lstlisting}
infix fun Int.times(i: Int) = this * i
val life = 6 times 7
                \end{lstlisting}
            \end{onlyenv}
        \end{exampleblock}
        \begin{exampleblock}
            <2->{StdLibInfixFunctions.kt}
            \begin{onlyenv}
                \begin{lstlisting}
val numberName = mapOf(
  1 to "one",
  2 to "two",
)
                \end{lstlisting}
            \end{onlyenv}
        \end{exampleblock}
    \end{frame}

    \begin{frame}[fragile]{Operators}
        Predefined set of operators in language like \texttt{+a, !a, a++, a..b, a + b, a in b} (and much more) that can be manually defined for every type
        \begin{exampleblock}{OperatorFunctions.kt}
            \begin{onlyenv}<1->
                \begin{lstlisting}
operator fun String.not() = this.map { 
  if (it.isUpperCase()) it.lowercaseChar()
  else it.uppercaseChar()
}.joinToString("")
                \end{lstlisting}
            \end{onlyenv}
            \begin{onlyenv}<2->
                \begin{lstlisting}
println(!"wOw")
                \end{lstlisting}
            \end{onlyenv}
        \end{exampleblock}
        \uncover<2->{and used as standard operators.}
    \end{frame}

    \begin{frame}[fragile]{High order functions}
        \begin{exampleblock}{HighOrderFunctions.kt}
            \begin{onlyenv}<1->
                \begin{lstlisting}
fun <T, U> Iterable<T>.foldLeft(
  initAcc: U, f: (U, T) -> U
): U {
  var acc = initAcc
  for (element in this) acc = f(acc, element)
  return acc
}
                \end{lstlisting}
            \end{onlyenv}
            \begin{onlyenv}<2->
                \begin{lstlisting}
numbers.foldLeft(0) { acc, e -> acc + e }
numbers.foldLeft(1, Int::times)
                \end{lstlisting}
            \end{onlyenv}
        \end{exampleblock}
        Idiomatic Kotlin code uses convention of having function parameters as last ones.
    \end{frame}

    \begin{frame}[fragile]{Generic functions}
        \begin{exampleblock}{GenericFunctions.kt}
            \begin{onlyenv}<1->
                \begin{lstlisting}
fun <T : Comparable<T>>
    List<T>.maxOr(default: () -> T): T =
  maxOrNull() ?: default()
                \end{lstlisting}
            \end{onlyenv}
        \end{exampleblock}
        while compiler is smart enough to usually guess all needed types
        \begin{exampleblock}
            <2->{GenericFunctions.kt}
            \begin{onlyenv}
                \begin{lstlisting}
listOf(3, 1, 4, 1, 5).maxOr { 92 }
                \end{lstlisting}
            \end{onlyenv}
        \end{exampleblock}
    \end{frame}

    \begin{frame}[fragile]{Inline functions}
        \begin{exampleblock}{InlineFunctions.kt}
            \begin{onlyenv}<1->
                \begin{lstlisting}
inline fun <reified T, V>
    Any?.letOrNull(f: (T) -> V) =
  if (this is T) f(this) else null
                \end{lstlisting}
            \end{onlyenv}
            \begin{onlyenv}<2->
                \begin{lstlisting}
"Kotlin".letOrNull<String, Int> { it.length }
                \end{lstlisting}
            \end{onlyenv}
            \begin{onlyenv}<3->
                \begin{lstlisting}
"Kotlin".letOrNull { str: String -> str.length }
                \end{lstlisting}
            \end{onlyenv}
        \end{exampleblock}
        where the function parameter can have also \texttt{crossinline} or \texttt{noinline} modifiers
    \end{frame}

    \begin{frame}[fragile]{Perform transformations on \texttt{it}}
        \begin{exampleblock}{ItMappers.kt}
            \begin{onlyenv}<1->
                \begin{lstlisting}
inline fun <T> T.also(f: (T) -> Unit): T {
  f(this)
  return this
}

inline fun <T, R> T.let(f: (T) -> R): R {
  return f(this)
}
                \end{lstlisting}
            \end{onlyenv}
        \end{exampleblock}
    \end{frame}

    \begin{frame}[fragile]{Perform transformations on \texttt{it} - examples}
        \begin{exampleblock}{ItMappers.kt}
            \begin{onlyenv}<1->
                \begin{lstlisting}
val maxVal = listOf(3, 1, 4)
  .maxOrNull()
  .also(::println)

val doubled = maxVal?.let { it * 2 }
                \end{lstlisting}
            \end{onlyenv}
        \end{exampleblock}
    \end{frame}

    \begin{frame}[fragile]{Perform transformations on \texttt{this}}
        \begin{exampleblock}{ThisMappers.kt}
            \begin{onlyenv}<1->
                \begin{lstlisting}
inline fun <T> T.apply(f: T.() -> Unit): T {
  this.f()
  return this
}
                \end{lstlisting}
            \end{onlyenv}
            \begin{onlyenv}<2->
                \begin{lstlisting}
val server = JettyServer().apply {
  install(ForceHttps)
  install(Cors)
  start()
}
                \end{lstlisting}
            \end{onlyenv}
        \end{exampleblock}
    \end{frame}

    \begin{frame}[fragile]{Tail recursion}
        \begin{exampleblock}{TailRecursiveFunctions.kt}
            \begin{onlyenv}<1-1>
                \begin{lstlisting}
tailrec fun factorial(c: UInt, r: UInt): UInt =
  if (c > 1U) factorial(c - 1U, r * c) else r
                \end{lstlisting}
            \end{onlyenv}
            \begin{onlyenv}<2->
                \begin{lstlisting}
fun factorial(n: UInt): UInt {
  tailrec fun go(c: UInt, r: UInt): UInt =
    if (c > 1U) go(c - 1U, r * c) else r
  return go(n, 1U)
}
                \end{lstlisting}
            \end{onlyenv}
        \end{exampleblock}
        Optimized by compiler to standard \texttt{while} loop pattern.
    \end{frame}

    \subsection{Classes}

    \begin{frame}{Classes in Kotlin}
        In Kotlin we can define classes similarly as in Java but also we can define:
        \begin{itemize}
            \pause
            \item multiple classes in single file \pause
            \item extension functions in classes \pause
            \item \texttt{sealed} classes as well as \texttt{enum} classes \pause
            \item \texttt{object}s being singleton classes \pause
            \item \texttt{typealias}es resolved at compile-time \pause
            \item inline classes as \texttt{value class}es \pause
            \item \texttt{inner class}es as well as nested \texttt{class}es
        \end{itemize}
    \end{frame}

    \begin{frame}[fragile]{Base classes}
        \begin{exampleblock}{BasicClasses.kt}
            \begin{onlyenv}<1->
                \begin{lstlisting}
abstract class Animal constructor(
  val color: Color, val owner: Person
) : Comparable<Animal> {
  enum class Color { RED, GREEN, BLUE }

  override fun compareTo(other: Animal) =
    color.compareTo(other.color)
}
                \end{lstlisting}
            \end{onlyenv}
        \end{exampleblock}
    \end{frame}

    \begin{frame}[fragile]{Data classes}
        \begin{exampleblock}{DataClasses.kt}
            \begin{onlyenv}<1->
                \begin{lstlisting}
data class Person(val name: String, val age: Int)

val alice = Person("Alice", 42)
val equal = alice == Person("Bob", 22)
val olderAlice = alice.copy(age = 66)
                \end{lstlisting}
            \end{onlyenv}
            \begin{onlyenv}<2-2>
                \begin{lstlisting}
val (name, age) = alice
println("Hello, I'm $name and I'm $age")
                \end{lstlisting}
            \end{onlyenv}
            \begin{onlyenv}<3->
                \begin{lstlisting}
with(alice) {
  println("Hello, I'm $name and I'm $age")
}
                \end{lstlisting}
            \end{onlyenv}
        \end{exampleblock}
    \end{frame}

    \begin{frame}[fragile]{Example: Interpreter classes structure}
        \begin{exampleblock}{InterpreterClasses.kt}
            \begin{onlyenv}<1->
                \begin{lstlisting}
typealias Ident = String
typealias Env = MutableMap<Ident, Int?>
typealias IntOp = (Int, Int) -> Int
interface Exp { fun eval(env: Env): Int? }

sealed class BinOp(val op: IntOp)
abstract class AddOp(op: IntOp) : BinOp(op)
abstract class MulOp(op: IntOp) : BinOp(op)
                \end{lstlisting}
            \end{onlyenv}
        \end{exampleblock}
    \end{frame}

    \begin{frame}[fragile]{Example: Interpreter classes structure}
        \begin{exampleblock}{InterpreterClasses.kt}
            \begin{onlyenv}<1->
                \begin{lstlisting}
object Plus : AddOp(Int::plus)
object Minus : AddOp(Int::minus)
object Times : MulOp(Int::times)
object Divide : MulOp(Int::div)
                \end{lstlisting}
            \end{onlyenv}
        \end{exampleblock}
    \end{frame}

    \begin{frame}[fragile]{Example: Interpreter classes structure}
        \begin{exampleblock}{InterpreterClasses.kt}
            \begin{onlyenv}<1->
                \begin{lstlisting}
data class BinExp(
  val l: Exp, val op: BinOp, val r: Exp
): Exp {
  override fun eval(env: Env): Int? {
    val lValue = l.eval(env) ?: return null
    val rValue = r.eval(env) ?: return null
    return op.op(lValue, rValue)
  }
}
                \end{lstlisting}
            \end{onlyenv}
        \end{exampleblock}
    \end{frame}

    \begin{frame}[fragile]{Example: Interpreter classes structure}
        \begin{exampleblock}{InterpreterClasses.kt}
            \begin{onlyenv}<1->
                \begin{lstlisting}
@JvmInline
value class NumExp(val value: Int): Exp {
  override fun eval(env: Env): Int = value
}

data class IdentExp(val i: Ident): Exp {
  override fun eval(env: Env): Int? = env[i]
}
                \end{lstlisting}
            \end{onlyenv}
        \end{exampleblock}
    \end{frame}

    \begin{frame}[fragile]{Example: Interpreter classes structure}
        \begin{exampleblock}{InterpreterClasses.kt}
            \begin{onlyenv}<1->
                \begin{lstlisting}
sealed class Stmt
class ExpStmt(val exp: Exp) : Stmt()
class AssStmt(val i: Ident, val e: Exp) :
  Stmt()
                \end{lstlisting}
            \end{onlyenv}
        \end{exampleblock}
    \end{frame}

    \begin{frame}[fragile]{Example: Interpreter classes structure}
        \begin{exampleblock}{InterpreterClasses.kt}
            \begin{onlyenv}<1->
                \begin{lstlisting}
fun step(env: Env, stmt: Stmt) = when(stmt) {
  is ExpStmt -> env.also {
    println(stmt.exp.eval(it))
  }
  is AssStmt -> env.apply {
    put(stmt.i, stmt.e.eval(env))
  }
}
                \end{lstlisting}
            \end{onlyenv}
        \end{exampleblock}
    \end{frame}

    \begin{frame}[fragile]{Example: Interpreter classes structure}
        \begin{exampleblock}{InterpreterClasses.kt}
            \begin{onlyenv}<1->
                \begin{lstlisting}
data class Program(
  private val stmts: MutableList<Stmt>
  = mutableListOf()
) {
  fun interpret(env: Env = mutableMapOf()) =
    stmts.fold(env, ::step)

  // we could stop there but...
                \end{lstlisting}
            \end{onlyenv}
        \end{exampleblock}
    \end{frame}

    \begin{frame}[fragile]{Example: Interpreter classes structure}
        \begin{exampleblock}{InterpreterClasses.kt}
            \begin{onlyenv}<1->
                \begin{lstlisting}
  // let's revisit functions knowledge
  infix fun Ident.eq(e: Exp) {
    stmts.add(AssStmt(this, e))
  }
  operator fun Exp.not() {
    stmts.add(ExpStmt(this))
  }
}
                \end{lstlisting}
            \end{onlyenv}
        \end{exampleblock}
    \end{frame}

    \begin{frame}[fragile]{Example: Interpreter classes structure}
        \begin{exampleblock}{InterpreterClasses.kt}
            \begin{onlyenv}<1->
                \begin{lstlisting}
operator fun Exp.plus(e: Exp) =
  BinExp(this, Plus, e)
operator fun Exp.times(e: Exp) = 
  BinExp(this, Times, e)

inline fun def(i: Ident) = IdentExp(i)
inline fun num(i: Int) = NumExp(i)
                \end{lstlisting}
            \end{onlyenv}
        \end{exampleblock}
    \end{frame}

    \begin{frame}[fragile]{Interpreter DSL usage example}
        \begin{exampleblock}{InterpreterClasses.kt}
            \begin{onlyenv}<1->
                \begin{lstlisting}
val program = Program().apply {
  "x" eq num(42)
  "y" eq num(24)
  "z" eq def("x") + def("y")
  !def("z")
  !(def("z") * def("u"))
}
program.interpret()
println(program)
                \end{lstlisting}
            \end{onlyenv}
        \end{exampleblock}
        \uncover<2->{It was less than 100 short lines of code}\uncover<3->{ but the effect is amazing.}
    \end{frame}

    \subsection{Idioms}

    \begin{frame}[fragile]{Destructuring declarations}
        \begin{exampleblock}{PairDestructuring.kt}
            \begin{onlyenv}<1->
                \begin{lstlisting}
val keyValue = "key" to Unit
val (k, v) = keyValue
                \end{lstlisting}
            \end{onlyenv}
            \begin{onlyenv}<2->
                \begin{lstlisting}
// which corresponds to 
val desugaredK = keyValue.component1()
val desugaredV = keyValue.component2()
                \end{lstlisting}
            \end{onlyenv}
        \end{exampleblock}
        \uncover<2->{where these fuctions can be defined as \texttt{operator}s and are also predefined e.g. for lists and \texttt{data class}es.}
    \end{frame}

    \begin{frame}[fragile]{Destructuring in lambdas}
        \begin{exampleblock}{LambdaDestructuring.kt}
            \begin{onlyenv}<1->
                \begin{lstlisting}
{ a -> ... } // one parameter
{ a, b -> ... } // two parameters
{ (a, b) -> ... } // a destructured pair
{ (a, b), c -> ... } // pair and parameter
                \end{lstlisting}
            \end{onlyenv}
        \end{exampleblock}
    \end{frame}

    \begin{frame}[fragile]{Delegated properties}
        Classes having defined \texttt{operator fun}ctions \texttt{getValue} and \texttt{setValue}
        \begin{exampleblock}{Observables.kt}
            \begin{onlyenv}<1->
                \begin{lstlisting}
import kotlin.properties.Delegates.observable

class User {
  var name: String by observable("noname") {
    _, old, new -> println("$old -> $new")
  }
}
                \end{lstlisting}
            \end{onlyenv}
        \end{exampleblock}
    \end{frame}

    \begin{frame}[fragile]{Delegated properties}
        The most commonly used is probably \texttt{lazy} initializer
        \begin{exampleblock}
            <1->{LazyProperties.kt}
            \begin{onlyenv}
                \begin{lstlisting}
class Book {
  val description: String by lazy {
    "value downloaded from database"
  }
}
                \end{lstlisting}
            \end{onlyenv}
        \end{exampleblock}
        which in other languages usually have to be defiend manually.
    \end{frame}

    \begin{frame}[fragile]{Ranges}
        There is only single range operator in language
        \begin{exampleblock}{Ranges.kt}
            \begin{onlyenv}<1->
                \begin{lstlisting}
val include = 1..42
val notInclude = 1 until 42
val down = 42 downTo 1
val collection = listOf(1, 2, 3)
for (i in collection.indices)
	println("At $i got ${collection[i]}")
                \end{lstlisting}
            \end{onlyenv}
            \begin{onlyenv}<2->
                \begin{lstlisting}
for ((i, v) in collection.withIndex())
	println("At $i got $v")
                \end{lstlisting}
            \end{onlyenv}
        \end{exampleblock}
    \end{frame}

    \begin{frame}[fragile]{"Not implemented" marker}
        \begin{exampleblock}{TODO.kt}
            \begin{onlyenv}<1->
                \begin{lstlisting}
fun `does P = NP`(): Boolean =
  TODO("Waiting for proof")
                \end{lstlisting}
            \end{onlyenv}
            \begin{onlyenv}<3->
                \begin{lstlisting}
inline fun TODO(reason: String): Nothing =
  throw NotImplementedError(
    "An operation is not implemented: $reason")
                \end{lstlisting}
            \end{onlyenv}
        \end{exampleblock}
        \uncover<2->{Not implemented function can have proper signature. It is part of standard library, not of the language.}
    \end{frame}

    \begin{frame}[fragile]{Tricky lambda scope}
        \begin{exampleblock}{Swap.kt}
            \begin{onlyenv}<1->
                \begin{lstlisting}
var a = 42
var b = 24
                \end{lstlisting}
            \end{onlyenv}
            \begin{onlyenv}<2-2>
                \begin{lstlisting}
val temp = a
a = b
b = temp
                \end{lstlisting}
            \end{onlyenv}
            \begin{onlyenv}<3->
                \begin{lstlisting}
a = b.also { b = a }
// now a = 24 and b = 42
                \end{lstlisting}
            \end{onlyenv}
        \end{exampleblock}
        \uncover<4->{as in Kotlin variables captured in the closure can be modified in the lambda. It is not used commonly as \texttt{var}s aren't used.}
    \end{frame}


    \section{Infrastructure}

    \subsection{Kotlin Script}

    \begin{frame}{Kotlin Scripts}
        \begin{itemize}
            \item Kotlin offers not only REPL for easy testing code but also provides mechanism of scripts \pause
            \item they are standard source files with extension \texttt{.kts} \pause
            \item can be run with \texttt{kotlinc} compiler but also custom solution like \texttt{kscript} exists \pause
            \item defines the structure of \texttt{Gradle} project
        \end{itemize}
    \end{frame}

    \begin{frame}[fragile]{Shell script}
        \begin{exampleblock}{SimpleScript.kts}
            \begin{onlyenv}<1->
                \begin{lstlisting}
#!/usr/bin/env kscript
import java.io.File

val dir = args.singleOrNull() ?: "."
                \end{lstlisting}
            \end{onlyenv}
            \begin{onlyenv}<2->
                \begin{lstlisting}
File(dir)
  .walkTopDown()
  .filter(File::isFile)
  .sorted()
  .forEach(::println)
                \end{lstlisting}
            \end{onlyenv}
        \end{exampleblock}
    \end{frame}

    \begin{frame}[fragile]{Kotlin DSL in Gradle}
        \begin{exampleblock}{build.gradle.kts}
            \begin{onlyenv}<1->
                \begin{lstlisting}
plugins {
  kotlin("jvm") version "1.5.31"
}

// ...

tasks.withType<KotlinCompile> {
  kotlinOptions.jvmTarget = "1.8"
}
                \end{lstlisting}
            \end{onlyenv}
        \end{exampleblock}
    \end{frame}

    \subsection{Kotlin Multiplatform}

    \begin{frame}{Kotlin Multiplatform}
        \begin{itemize}
            \item can be compiled to native executables, JavaScript files and also to WebAssembly \pause
            \item features code sharing between all or some platforms used in project \pause
            \item uses mechanism of \texttt{expect} and \texttt{actual} declarations \pause
            \item reuse the multiplatform logic in common and platform-specific code (it's even
            simpler when there are more and more multiplatform libraries)
        \end{itemize}
    \end{frame}

    \begin{frame}[fragile]{\texttt{expect} and \texttt{actual} in action}
        \begin{exampleblock}{Common.kts}
            \begin{onlyenv}<1->
                \begin{lstlisting}
// common
expect fun randomUUID(): String
                \end{lstlisting}
            \end{onlyenv}
            \begin{onlyenv}<2->
                \begin{lstlisting}
// Android
actual fun randomUUID() = 
  java.util.UUID.randomUUID().toString()
                \end{lstlisting}
            \end{onlyenv}
            \begin{onlyenv}<3->
                \begin{lstlisting}
// iOS
actual fun randomUUID(): String = 
  platform.Foundation.NSUUID().UUIDString()
                \end{lstlisting}
            \end{onlyenv}
        \end{exampleblock}
    \end{frame}

    \begin{frame}{Multi languages interoperability}
        Kotlin Native uses LLVM backend for compiling to native binaries. It can create:
        \begin{itemize}
            \item executables for many platforms \pause
            \item a static library or dynamic library with C headers for C/C++ projects \pause
            \item an Apple framework for Swift and Objective-C projects \pause
        \end{itemize}
        Kotlin supports using libraries from:
        \begin{itemize}
            \item static and dynamic C libraries \pause
            \item Swift and Objective-C frameworks \pause
        \end{itemize}
        Let's try CLion or AppCode to test these great features if you're interested.
    \end{frame}

    \begin{frame}{Kotlin Multiplatform project structure}
        Project defines hirarchical structure of sourcesets, for which the dependency relation is defined.
        E.g. we can have:
        \begin{itemize}
            \item \texttt{commonMain} source set containing all common logic \pause
            \item \texttt{jvmMain:commonMain} code specific for JVM \pause
            \item \texttt{jsMain:commonMain} code specific for web \pause
            \item \texttt{desktopMain:commonMain} code specific for all desktop platforms\pause
            \begin{itemize}
                \item \texttt{linuxX64Main:desktopMain} code specific for Linux x64 platforms \pause
                \item \texttt{macosX64Main:desktopMain} code specific for MacOS x64 platforms
            \end{itemize}
        \end{itemize}
    \end{frame}

    \begin{frame}{Kotlin new compiler - K2}
        \begin{itemize}
            \item Kotlin started with no native backend and no coroutines \pause
            \item introduce new type inference algorithm \pause
            \item provide new JVM and JS IR backend \pause
            \item introducing FIR (frontend intermediate representation) \pause
            \item goal: performance improvements, provide an API for compiler extensions
        \end{itemize}
    \end{frame}

    \section*{Appendix}
    \subsection*{Recommendation}

    \begin{frame}{Kotlin in Action - highly recommend}
        \centering
        \includegraphics[width=140pt]{images/action.png}
    \end{frame}

    {
        \setbeamertemplate{logo}{}
        \setbeamertemplate{headline}{}
        \begin{frame}{}
            \centering
            \includegraphics[width=100pt]{images/blink.png}\\\vspace{1cm}
            \emph{\Large Thank you for your attention}
        \end{frame}
    }

\end{document}
